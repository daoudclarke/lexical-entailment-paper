\documentclass[11pt]{article}
\usepackage{eacl2014}
\usepackage{times}
\usepackage{url}
\usepackage{latexsym}
\usepackage{graphicx}
\setlength\titlebox{6.5cm}

\newcommand\entBLESS{\mbox{\em hyponym\/}_{\mbox{\scriptsize BLESS}}}
\newcommand\coordBLESS{\mbox{\em cohyponym\/}_{\mbox{\scriptsize BLESS}}}
\newcommand\entWN{\mbox{\em hyponym\/}_{\mbox{\scriptsize WN}}}
\newcommand\coordWN{\mbox{\em cohyponym\/}_{\mbox{\scriptsize WN}}}
\newcommand\invCL[1]{\mbox{\em invCL\/}(#1)}
\newcommand\simcos[1]{\mbox{\em sim\/}_{\mbox{\em cos}}(#1)}
\newcommand\simlin[1]{\mbox{\em sim\/}_{\mbox{\em lin}}(#1)}
\newcommand\width[1]{\mbox{\em width\/}(#1)}
\newcommand\rel[1]{\mbox{\em rel\/}(#1)}
\newcommand\pair[2]{$\langle$\texttt{#1} ISA \texttt{#2}$\rangle$}

\title{Learning to Distinguish Hypernyms and Co-Hyponyms}
%\author{Julie Weeds, Daoud Clarke, Jeremy Reffin, Bill Keller and David Weir\\ Department of Informatics, University of Sussex, Brighton, UK}

\author{First Author \\
  Affiliation / Address \\
  {\tt email@domain} \\\And
  Second Author \\
  Affiliation / Address \\
  {\tt email@domain} \\}

\date{} % Activate to display a given date or no date (if empty),
         % otherwise the current date is printed 

\begin{document}

\maketitle

\begin{abstract}
This work is concerned with distinguishing different semantic relations which exist between distributionally similar words.  We compare a novel approach based on training a linear Support Vector Machine on pairs of feature vectors with state-of-the-art methods based on distributional similarity. We show that the new supervised approach does better even when there is minimal information about the target words in the training data, giving a 15\% reduction in error rate over unsupervised approaches.

%These results should lead to further improvements in the fields of automatic thesaurus generation, lexical and textual entailment and compositional distributional semantics.
\end{abstract}
\section{Introduction}

Over recent years there has been much interest in the field of distributional semantics, drawing on the distributional hypothesis: words that occur in similar contexts tend to have similar meanings \cite{Harris1954}.   There is a large body of work on the use of different similarity measures \cite{Lee1999,Weeds2003,Curran2004} and many researchers have built thesauri (i.e.~lists of ``nearest neighbours'') automatically and applied them in a variety of applications, generally with a good deal of success.

In early research there was much interest in how these automatically generated thesauri compare with human-constructed gold standards such as WordNet and Roget \cite{Lin1998,Kilgarriff2000}.  More recently, the focus has tended to shift to building thesauri to alleviate the sparse-data problem.  Distributional thesauri have been used in a wide variety of areas including sentiment classification~\cite{Bollegala2011}, WSD~\cite{miller-EtAl:2012:PAPERS,khapra-EtAl:2010:ACL}, PP attachment~\cite{Calvo05distributionalthesaurus}, automatic confusion set generation~\cite{xue-hwa:2012:PAPERS}, textual entailment~\cite{berant-dagan-goldberger:2010:ACL}, co-reference resolution~\cite{lee-EtAl:2012:EMNLP-CoNLL}, predicting semantic compositionality~\cite{bergsma-EtAl:2010:EMNLP}, acquisition of semantic lexicons~\cite{mcintosh:2010:EMNLP}, conversation entailment~\cite{zhang-chai:2010:EMNLP}, semantic role classification~\cite{zapirain-EtAl:2010:NAACLHLT}, lexical substitution~\cite{szarvas-biemann-gurevych:2013:NAACL-HLT}, taxonomy induction~\cite{fountain-lapata:2012:NAACL-HLT}, detection of visual text~\cite{dodge-EtAl:2012:NAACL-HLT}, and parser lexicalisation~\cite{rei-briscoe:2013:NAACL-HLT}.  

A primary focus of distributional semantics has been on identifying words which are similar to each other. However, semantic similarity encompasses a variety of different lexico-semantic and topical relations.  Even if we just consider nouns, an automatically generated thesaurus will tend to return a mix of synonyms, antonyms, hyponyms, hypernyms, co-hyponyms, meronyms and other topically related words.  A central problem here is that whilst most measures of distributional similarity are symmetric, some of the important semantic relations are not.  The hyponymy relation (and converse hypernymy) which forms the ISA backbone of taxonomies and ontologies such as WordNet~\cite{Fellbaum:98}, and determines lexical entailment \cite{Geffet2005}, is asymmetric. On the other hand, the co-hyponomy relation which relates two words unrelated by hyponomy but sharing a (close) hypernym, is symmetric, as are synonymy and antonymy. Table \ref{table:neighbours} shows the distributionally nearest neighbours of the words \texttt{cat}, \texttt{animal} and \texttt{dog}. %using the Byblo software {\bf comment: reference??} and Wikipedia.  
In the list for \texttt{cat} we can see 2 hypernyms and 13 co-hyponyms\footnote{We read \texttt{cat} in the sense \texttt{domestic cat} rather than \texttt{big cat}, hence \texttt{tiger} is a co-hyponym rather than hyponym of \texttt{cat}}. 

\begin{table}[th]
\begin{tabular}{|c|c|c|}
\hline
cat&animal&dog\\
\hline
dog 0.32&bird 0.36&cat 0.32\\
animal 0.29&fish 0.34&animal 0.31\\
rabbit 0.27&creature 0.33&horse 0.29\\
bird 0.26&dog 0.31&bird 0.26\\
bear 0.26&horse 0.30&rabbit 0.26\\
monkey 0.26&insect 0.30&pig 0.25\\
mouse 0.25&species 0.29&bear 0.26\\
pig 0.25&cat 0.29&man 0.25\\
snake 0.24&human 0.28&fish 0.24\\
horse 0.24&mammal 0.28&boy 0.24\\
rat 0.24&cattle 0.27&creature 0.24\\
elephant 0.23&snake 0.27&monkey 0.24\\
tiger 0.23&pig 0.26&snake 0.24\\
deer 0.23&rabbit 0.26&mouse 0.24\\
creature 0.23&elephant 0.25&rat 0.23\\
\hline
\end{tabular}
\label{table:neighbours}
\caption{Top 15 neighbours of \texttt{cat}, \texttt{animal} and \texttt{dog} generated using Lin's similarity measure \cite{Lin1998} considering all words and dependency features occurring 100 or more times in Wikipedia}
\end{table}

Distributional similarity is being deployed (e.g., \newcite{Dinu2012}) in situations where it can be useful to be able to distinguish between these different relationships. Consider the following two sentences.
\begin{equation}
\label{sentA}
\mbox{\em The cat ran across the road.}
\end{equation}
\begin{equation}
\label{sentB}
\mbox{\em The animal ran across the road.}
\end{equation}
Sentence~\ref{sentA} textually entails sentence~\ref{sentB}, but sentence~\ref{sentB} does not textually entail sentence~\ref{sentA}. The ability to determine whether entailment holds between the sentences, and in which direction, depends on the ability to identify hyponymy.   Given a similarity score of 0.29 between \texttt{cat} and \texttt{animal}, how do we know which is the hyponym and which is the hypernym? 

In applying distributional semantics to the problem of textual entailment, there is a need to generalise lexical entailment to the phrasal and sentence level. Thus, the ability to distinguish different semantic relations is crucial if approaches to the composition of distributional representations of meaning that are currently receiving considerable interest~\cite{Widdows:08,Mitchell:08,Baroni2010,Grefenstette:11,Socher:12} are to be applied to the textual entailment problem.

We formulate the challenge as follows: Consider a set of pairs of similar words $\langle A,B\rangle$ where one of three relationships hold between $A$ and $B$: $A$ lexically entails $B$, $B$ lexically entails $A$ or  $A$ and $B$ are related by co-hyponymy. Given such a set, how can we determine which relationship holds? In Section \ref{sect:relwork}, we discuss existing attempts to address this problem through the use of various \emph{directional} measures of distributional similarity.  This paper  considers the effectiveness of various supervised approaches, and makes the following contributions:\\[2pt]
(1)~We show that a SVM can distinguish the entailment and co-hyponmy relations, achieving a significant reduction in error rate in comparison to  existing state-of-the-art methods based on the notion of distributional generality.\\[2pt]
(2)~By comparing two  different data sets, one built from  BLESS~\cite{Baroni2011} and the other from WordNet~\cite{Fellbaum:98}, we derive important insights into the requirements of a valid evaluation of supervised approaches, and provide a data set for further research in this area.\\[2pt]
(3)~We show that when learning how to determine an ontological relationship between a pair of similar words by means of the word's distributional vectors, quite different vector operations are useful when identifying different ontological relationships. In particular, the using the difference between the vectors for pair of words is appropriate for the entailment task, whereas adding the vectors works well for the co-hyponym task.

\section{Related Work}
\label{sect:relwork}

\newcite{Lee1999} first noted that the substitutability of one word for another was asymmetric and proposed the alpha-skew divergence measure, an asymmetric version of the Kullback-Leibler divergence measure.  She found that this measure improved results in language modelling, when a word's distribution is smoothed using the distributions of its nearest neighbours.

\newcite{Weeds2004} proposed a notion of distributional generality, observing that more general words tend to occur in a larger variety of contexts than more specific words.  For example, we would expect to be able to replace any occurrence of \texttt{cat} with \texttt{animal} and so all of the contexts of \texttt{cat} must be plausible contexts for \texttt{animal}.  However, not all of the contexts of \texttt{animal} would be plausible for \texttt{cat}, e.g., ``the monstrous animal barked at the intruder''.  \newcite{Weeds2004} attempt to capture this asymmetry by framing word similarity in terms of co-occurrence retrieval~\cite{Weeds2003}, where precision and recall are defined as:

\[
P_{ww}(u,v) = \frac{\Sigma_{f \in F(u) \cap F(v)} I(u,f)}{\Sigma_{f \in F(u)} I(u,f)}
\]

\[
R_{ww}(u,v) = \frac{\Sigma_{f \in F(u) \cap F(v)} I(v,f)}{\Sigma_{f \in F(v)} I(v,f)}
\]
where $I(n,f)$ is the  pointwise mutual information (PMI) between noun $n$ and feature $f$ and F(n) is the set of all features $f$ for which $I(n,f)>0$

By comparing the precision and recall of one word's retrieval of another word's contexts, they were able to successfully identify the direction of an entailment relation in 71\% of pairs drawn from WordNet.  However, this was not significantly better than a baseline which proposed that the most frequent word was the most general.

\newcite{Clarke:09} formalised the idea of distributional generality using a partially ordered vector space.  He also argued for using a variation of co-occurrence retrieval where precision and recall are defined as:

\[
P_{cl}(u,v) = \frac{\Sigma_{f \in F(u) \cap F(v)} min(I(u,f),I(v,f))}{\Sigma_{f \in F(u)} I(u,f)}
\]

\[
R_{cl}(u,v) = \frac{\Sigma_{f \in F(u) \cap F(v)} min(I(u,f),I(v,f))}{\Sigma_{f \in F(v)} I(v,f)}
\]

\newcite{Lenci2012} took the notion further and hypothesised that more general terms should have high recall and low precision, which would thus make it possible to distinguish them from other related terms such as synonyms and co-hyponyms.  They proposed a variant of the \newcite{Clarke:09} measure to identify hypernyms:

\[
invCL(u,v) = \sqrt[2]{P_{cl}(u,v)*(1-R_{cl}(u,v))}
\]
Evaluation on the BLESS data set~\cite{Baroni2011}, showed that this measure is  better at distinguishing hypernyms from other semantic relations than the measures of \newcite{Weeds2004} and \newcite{Clarke:09}.

\newcite{Geffet2005} proposed an approach based on \emph{feature inclusion}, which extends the rationale of~\newcite{Weeds2004} to lexical entailment. Using data from the web they demonstrated a strong correlation between complete inclusion of prominent features and lexical entailment. However, they were unable to assess this using an off-line corpus due to data sparseness.

\newcite{Szpektor2008} found that the $P_{ww}$ measure tends to promote relationships between infrequent words with narrow vectors (i.e.~those with relatively few distinct context features).  They proposed using the geometric average of $P_{ww}$ and the symmetric similarity measure of \newcite{Lin1998} in order to penalise low frequency words.   

\newcite{Kotlerman2010} apply the IR evaluation method of \emph{Average Precision} to the problem of identifying lexical inference and use the balancing approach of \newcite{Szpektor2008} to demote similarities for narrow feature vectors; their measure is called \emph{balAPinc}.  They show that all of the asymmetric similarity measures previously proposed perform much better than symmetric similarity measures on a directionality detection experiment, and that their method and that of \newcite{Clarke:09} outperform the others with statistical significance. They also show that their measure is superior when used for term expansion in an event detection task.

The Stanford WordNet project \cite{Snow:04} expands the WordNet
taxonomy by analysing large corpora to find patterns that are
indicative of hyponymy. For example, the pattern ``$\mathit{NP}_X$ and
other $\mathit{NP}_Y$'' is an indication that $\mathit{NP}_X$ is a
$\mathit{NP}_Y$, i.e.~that $\mathit{NP}_X$ is a hyponym of
$\mathit{NP}_Y$. They use machine learning to identify other such
patterns from known hyponym-hypernym pairs, and then use these
patterns to find new relations in the corpus. The
transitivity relation of the taxonomy is enforced by searching only over valid
taxonomies and evaluating the likelihood of each taxonomy given the
available evidence~\cite{Snow:06}. The approach is similar to
ours in providing a supervised method of learning semantic relations, but relies on having features for occurrences of pairs of terms rather than just vectors for terms themselves. Our approach is therefore more generally applicable to systems which compose distributional
representations of meaning.

Most recently, \newcite{rei-briscoe:2013:NAACL-HLT} note that hyponyms are well suited for lexical substitution.  In their experiments with smoothing edge scores for parser lexicalisation, they find that a directional similarity measure,  \emph{WeightedCosine}\footnote{The details of this measure are unpublished.}, performs best. Also of note, \newcite{Mikolov2013} propose a vector offset method to capture syntactic and semantic regularities between word representations learnt by a recurrent neural network language model.  \newcite{yih-zweig-platt:2012:EMNLP-CoNLL} present a method for distinguishing synonyms and antonyms by inducing polarity in a document-term matrix before applying Latent Semantic Analysis.


\section{Methodology}

\subsection{Vector Representations}
\label{sect:representations}

Our goal is to build classifiers that establish whether or not a given
semantic relation, \emph{rel}, holds between two similar words $A$ and
$B$.  Support vector machines (SVMs), which are effective across a
variety of classification scenarios, learn a boundary between two
classes from a set of positive and negative example vectors.  The two
classes correspond to the relation \emph{rel} holding or not holding.
Here, however, we do not start with a single vector, but with two
distributional vectors $v_A$ and $v_B$ for the words $A$ and $B$,
respectively. These vectors must be combined in some way to produce
the SVM's input, and a number of ways were considered, defined in
table \ref{table:classifiers}.

Of these measures, the vector difference (\emph{svm\-DIFF} and
\emph{knnDIFF}) and direct sum (\emph{svmCAT}) are asymmetric, whereas
the sum and pointwise multiplication (\emph{svmADD} and \emph{svmMULT})
are symmetric. Note that these methods behave in very different ways,
for example, vector difference produces smaller vectors when given
similar vectors, whereas pointwise multiplication produces larger
vectors when the inputs are similar.

% Vector difference produces small vectors when given
% similar vectors. It does not distinguish cases where two
% words share a feature with similar high weights from cases where
% neither word has the feature strongly, and it emphasises features
% which one word has strongly but the other does not.

% Pointwise multiplication finds the intersection of the two vectors:
% the $i$th component is zero for every non-shared feature $i$.
% Dissimilar vectors will have small multiplication vectors and similar
% vectors will have large multiplication vectors.

% Vector addition emphasises the intersection of the vectors whilst
% retaining information about all features in the union of the two
% vectors.

% The direct sum retains all of the information from both vectors
% including the direction.

Distributional information was collected for all of the nouns from Wikipedia provided they had occurred $100$ or more times.  The features used were grammatical dependency features (also occurring $100$ or more times) extracted using the MALT parser~\cite{Nivre2006}.  The value of each feature is the (positive) PMI between the noun and the feature. The total number of noun vectors which can be harvested from Wikipedia with these parameters is $124,345$.  Random $1000$ dimensional vectors were also constructed in order to be able to assess the impact of using distributional information over a system which simply learns the implied taxonomy in the training data and applies it to the test data.  The dimensionality of the random vectors was chosen to be $1000$ since this substantially exceeds the average number ($398$) of non-zero features in the Wikipedia vectors.

\subsection{Classifiers}

We constructed linear SVMs for each of the vector operations outlined in Section \ref{sect:representations}.  We used linear SVMs for speed and simplicity, since the point is to compare the different vector representations of the pairings.  For comparison, we also constructed a number supervised, unsupervised, and  weakly supervised classifiers. These are listed in Table~\ref{table:classifiers}.  



\begin{table*}[ht]
\centering
\begin{tabular}{|l|p{14cm}|}
\hline
\emph{svmDIFF}& A linear SVM trained on the vector difference $v_B - v_A$\\
\emph{svmMULT}& A linear SVM trained on the pointwise product vector $v_B * v_A$\\
\emph{svmADD}& A linear SVM trained on the vector sum $v_B + v_A$\\
\emph{svmCAT}& A linear SVM trained on the vector concatenation $v_B\oplus v_A$\\
\emph{svmSING}& A linear SVM trained on the vector $v_B$\\
\emph{knnDIFF}& k nearest neighbours (knn) trained on the vector difference $v_B - v_A$.$1< k<50$\\
\hline
\emph{widthdiff}& $\width{B} > \width{A} \rightarrow\rel{A,B}$  where $\width{A}$ is  number of non-zero features in $A$\\
\emph{singlewidth}& $\width{B} > p \rightarrow \rel{A,B}$\\
\emph{cosineP}& $\simcos{A,B} > p \rightarrow\rel{A,B}$ where $\simcos{A,B}$ is cosine similarity using PPMI\\
\emph{linP}&$\simlin{A,B} > p \rightarrow\rel{A,B}$ \cite{Lin1998}\\
\emph{CRdiff}& $P_{ww}(A,B) > R_{ww}(A,B) \rightarrow\rel{A,B}$  \cite{Weeds2004}\\
\emph{clarkediff}&$P_{cl}(A,B) > R_{cl}(A,B) \rightarrow\rel{A,B}$  \cite{Clarke:09}\\
\emph{invCLP}&$\invCL{A,B} > p \rightarrow\rel{A,B}$  \cite{Lenci2012}\\
\emph{balAPinc}&Balanced average precision \cite{Kotlerman2010}\\
\emph{most freq}&The most frequent label in the training data is assigned to every test point.\\
\hline
\end{tabular}
\caption{Implemented classifiers}
\label{table:classifiers}
\end{table*}

\subsection{Data Sets}
\label{sect:data}

One of key the challenges of this work has been to construct a data set which accurately and validly tests our hypotheses.  Such a data set needs to be balanced in many respects in order to prevent the supervised classifiers making use of artefacts of the data.  This would not only make it unfair to compare the supervised approaches with the unsupervised approaches, but also make it unlikely that our results would be generalisable to other data.  Here, we outline the requirements for the data sets, the importance of which is demonstrated by our initial results for a data set which does not satisfy all of them.

There should be an equal number of positive and negative examples of a semantic relation.  Thus, random guessing or labelling with the most frequently seen label in the training data will yield 50\% accuracy and precision.  An advantage of incorporating this requirement means that evaluation can be in terms of simple accuracy (or error rate).

It should not be possible to do well simply by considering the distributional similarity of the terms.  Hence, the negative examples need to be pairs of equally similar words, but where the relationship under consideration does not hold.

It should not be possible to do well by pre-supposing an entailment relation and guessing the direction.  For example, it has been shown~\cite{Weeds2004} that given a pair of entailing words selected from WordNet, over 70\% of the time the more frequent word is also the entailed word.

It should not be possible to do well using ontological information learnt about one or both of the words from the training data that is not generalisable to their distributional representations.  For example, it should not be possible for the classifier simply to learn directly from the training pairs \pair{cat}{mammal} and \pair{mammal}{animal}  that \pair{cat}{animal}.  Furthermore, we must ensure that a classifier cannot learn that a particular word is near the top of the ontological hierarchy, and, as a result, do well by guessing that a particular pairing probably has an entailment relation.  For example, given many pairs such as \pair{cat}{animal}, \pair{dog}{animal}, a system which guessed \pair{rabbit}{animal} but not \pair{animal}{rabbit} would do better than random guessing.  Whilst both of these types of information could be useful in a hybrid system, they do not require any distributional information and therefore we would not be learning anything about the distributional features of \texttt{animal} which make it likely to be a hypernym.

\subsubsection{BLESS}

We have constructed two  data sets from BLESS~\cite{Baroni2011} which is a collection of examples of hypernyms, co-hyponyms, meronyms and random unrelated words for each of $200$ concrete, largely monosemous nouns. We will refer to these $200$ nouns as the BLESS concepts.

\smallskip

\noindent $\entBLESS$ is a set of $1976$ labelled pairs of nouns.  For each BLESS concept, 80\% of the hypernyms were randomly selected to provide positive examples of entailment.  The remaining hypernyms for the given concept were reversed and taken with the same number of co-hyponyms, meronyms and random words to form negative examples of entailment.  A filter was applied to ensure that duplicate pairs were not included (e.g., if $\langle\texttt{cat},\texttt{animal}\rangle$ is a positive pair then $\langle\texttt{animal},\texttt{cat}\rangle$ cannot be a negative pair). 

\smallskip

\noindent $\coordBLESS$ is a set of 5835 labelled pairs of nouns.  For each BLESS concept, the co-hyponyms were taken as positive examples of this relation.  The same total number of (and split evenly between) hypernyms, meronyms and random words was taken to form the negative examples.  The order of 50\% of the pairs was reversed and again duplicate pairs were disallowed. 

\smallskip

In both cases the pairs are labelled as positive or negative for the specified semantic relation and in both cases there are equal ($\pm 1$) numbers of positive and negative examples.  For 99\% of the generated BLESS pairs, both nouns had associated vectors harvested from Wikipedia.  If a noun does not have an associated vector, the classifiers uses a zero vector.  

\subsubsection{WordNet}

We  constructed two data sets using WordNet.  Whilst these data sets are similar in size to the BLESS data sets they more adequately satisfy the requirements laid out above\footnote{Note that imposing these requirements on the BLESS data sets would lead to very small data sets, since information is only provided for $200$ nouns.}.  We constructed a list of all non-rare, largely monosemous, single word terms in WordNet.  To be considered non-rare, a word needed to have occurred in SemCor at least once (i.e.~frequency information is provided about it in the WordNet package) and to have occurred in WikiPedia at least $100$ times.  To be considered largely monosemous, the predominant sense of the word needed to account for over 50\% of the occurrences in the SemCor frequency information provided with WordNet.  This led to a list of $7613$ words. 

\smallskip

\noindent $\entWN$ is a set of $2564$ labelled pairs of nouns.  Pairs $\langle A,B\rangle$ were found in the list where $B$ is an ancestor of $A$ (i.e., $A$ lexically entails $B$).  Each found pair is added either as a positive or a negative in the ratio 2:1 provided that the reverse pairing has not already been added and provided that each word has not previously been used in that position.  Co-hyponym pairs (i.e., words which share a direct hypernym) were also found within the list.  Each found pair is added to the data set (as a negative) provided the reverse pairing has not already been added, and provided that neither word has already been seen in that position in a pairing (either in the entailment pairs or the co-hyponym pairs).  The same number of co-hyponym pairs as hypernym-hyponym negatives is selected. 

\smallskip

\noindent $\coordWN$ is a set of $3771$ labelled pairs of nouns.  It was constructed in the same way as $\entWN$ except the same number of co-hyponym pairs were selected as the total number of entailment pairs (in either direction).  These co-hyponym pairs were labelled as positive and the entailment pairs were labelled as negative.

\smallskip

In both these sets, the average path distance between entailment pairs is $1.64$, whereas path distance between co-hyponym pairs is $2$.

\subsection{Experimental Setup}

Most of our experiments were carried out using an implementation of
five-fold cross-validation using each combination of data set, vector
set and classifier.  In this setup, the pairs are randomly partitioned
into five subsets, one subset is held out for testing whilst the
classifiers are trained on the remaining four, and this process is
repeated using each subset as the test set.

For the BLESS datasets, we found in initial experiments that using
random vectors did not produce significantly different results from
using the vectors from Wikipedia, indicating that the classifiers were
learning ontological information implicit in the training data. In
order to address this, we removed any pair from the training data if
either word was present in the test data. In order to preserve a
reasonable amount of training data, we implemented this approach with
ten-fold cross-validation.

\section{Results}
\label{sect:results}

\begin{table}[ht]
\centering
\begin{tabular}{|l|c|c|}
\hline
Classifier&Hyponym&Cohyponym\\
\hline
svmDIFF&\textbf{0.74}&0.62\\
svmMULT&0.56&0.39\\
svmADD&0.66&0.41\\
svmCAT&0.68&0.40\\
svmSING&\textbf{0.75}&0.40\\
knnDIFF&0.54&0.58\\
\hline
most freq&0.54&0.61\\
cosineP&0.53&\textbf{0.79}\\
widthdiff&0.56&--\\
singlewidth&0.58&--\\
invCLP&0.54&--\\
balAPinc&0.54&--\\
\hline
\end{tabular}
\caption{Accuracy Figures for the $\entBLESS$ and
  $\coordBLESS$ data sets (standard errors $<$ 0.02), for vectors generated
  from Wikipedia (Wk) and random vectors (Rd). For
  $\coordBLESS$, results for measures designed to detect hyponymy
  have been omitted.}
\label{table:results_ent}
\end{table}


Table~\ref{table:results_ent} summarises results for the $\entBLESS$ data set and the $\coordBLESS$ data set, and Table~\ref{table:results_WN} summarises results for the $\entWN$ data sets and the $\coordWN$ data sets.  We compare average accuracy for a number of different classifiers using $2$ different vector representations (in each of the experimental setups for the BLESS data).  In all cases, standard error was less than $0.02$.  For brevity we omit some of the classifiers which performed worse than (or not significantly differently to) other similar classifiers.  In particular, cosineP generally performed marginally better than linP; CRdiff and clarkediff were indistinguishable in performance and generally performed similarly to invCLP and widthdiff.

On the entailment task, the SVM methods do generally outperform the unsupervised methods.  However, the best performing model is svmSING, suggesting that, for this data set, it is best to try to learn the distributional features of more general terms, rather than comparing the vector representations of the two terms under consideration.   

On the coordinate task, the best performing classifier is the cosine measure. The poor performance of the SVM methods here can perhaps be explained by the paucity of the training data in this experimental set up with this data set.  If, for example, our test concept is \texttt{robin}, under the \texttt{holdback2} approach, we will not have any training pairs containing \texttt{robin}, or any training pairs containing any of the words to which \texttt{robin} is related in the data set.  It is this that motivated our approach to the construction of the WordNet data sets. In particular, we require that each word only be present once in each position of a pair --- allowing us to remove the implicit ontological information from the training set, without removing all knowledge of the distributional features of words in the target domain.

\begin{table}[ht]
\centering
\begin{tabular}{|l|c|c|c|c|}
\hline
&\multicolumn{2}{|c|}{$\entWN$}&\multicolumn{2}{|c|}{$\coordWN$}\\
\hline
Classifier&Wk&Rd&Wk&Rd\\
\hline
svmDIFF&\textbf{0.75}&0.53&0.37&0.45\\
svmMULT&0.45&0.46&0.60&0.54\\
svmADD&0.37&0.45&\textbf{0.68}&0.54\\
svmCAT&\textbf{0.74}&0.48&\textbf{0.64}&0.52\\
svmSING&0.69&0.49&0.58&0.49\\
knnDIFF&0.50&0.49&0.50&0.50\\
\hline
most freq&0.50&0.50&0.50&0.50\\
cosineP&0.53&0.50&0.50&0.50\\
widthdiff&0.70&0.50&--&--\\
singlewidth&0.65&0.50&--&--\\
clarkediff&0.70&0.50&--&--\\
invCLP&0.66&0.48&--&--\\
balAPinc&0.54&&--&--\\
\hline
\end{tabular}
\caption{Accuracy Figures for $\entWN$ and $\coordWN$ (standard errors $<$
  0.02), for vectors generated
  from Wikipedia (Wk) and random vectors (Rd).}
\label{table:results_WN}
\end{table}

Looking at the results for the $\entWN$ data set in Table \ref{table:results_WN}, the directional SVM methods (svmDIFF and svmCAT)  substantially outperform the symmetric SVM methods, and their performance is significantly better (at the 0.01\% level) than the unsupervised methods.  Also of note is the substantial difference between svmDIFF and knnDIFF.  Both of these methods are trained on the differences of vectors.  However, the linear SVM outperforms kNN by 19--25\%.  This may suggest that the shape of the vector space inhabited by the positive entailment pairs is particularly conducive for learning a linear SVM.  Positive and negative pairs are close together (as evidenced by the poor performance of kNN), but generally linearly separable.

Looking at the results for the $\coordWN$ data set, it is clear that the unsupervised methods cannot distinguish the co-hyponym pairs from the entailing pairs.  The supervised SVM methods do substantially better, with the best performance achieved by svmADD and svmCAT.  Both of these methods essentially retain information about all of the features of both words.  svmMULT does much better than svmDIFF, which suggests that the shared features are more indicative than the non-shared features for this task. 

The reasonably high performance of svmSING on both data sets suggests that words which have co-hyponyms in the data set tend to inhabit a somewhat different part of the feature space to words which are included as entailed words in the data set.  We hypothesise that there are specific features which more general words tend to share (regardless of their topic) which makes it possible to identify more general words from more specific words.  This phenomena is also observed in the BLESS data set which suggests that it is not simply an artefact of this particular data.

A final point to note is that svmADD and svmMULT  do significantly better than random with random vectors with the $\coordWN$ data set.  This suggests that the supervised methods are able to benefit slightly from some residual ontological information about the target words present in the training set.  For example, a decision about whether $\langle\texttt{cat},\texttt{dog}\rangle$ is a valid co-hyponym pair may be influenced by whether we have seen $\langle\texttt{mouse},\texttt{cat}\rangle$ as a valid co-hyponym pair, since this may suggest that cat is at the level in the hierarchy where lots of co-hyponyms tend to live.  However, all of the methods do vastly better with the distributional vectors harvested from Wikipedia, from which we conclude that the distributional information is much more significant than this residual ontological information.

\section{Conclusions and Further Work}

We have shown that it is possible to predict whether
or not there is a specific semantic relation between two words given
their distributional vectors, using a supervised approach based on linear SVMs. The   increase in accuracy over unsupervised methods is significant at the 0.01\% level and corresponds to a substantial reduction in error rate (over 15\%).

We have also shown that the choice of vector operation is significant.  Whilst concatenating the vectors, and therefore retaining all of the information from both vectors including direction, generally performs well, we have also shown that different vector operations are useful in establishing different relationships.  In particular, the vector difference operation which loses information about the original vectors, achieved performance indistinguishable from concatenation on the entailment task.  The addition operation, which also loses information, outperformed concatenation by 4\% (which is statistically significant at the 0.01\% level) on the coordinate task.

We have also shown that it is possible to outperform  state-of-the-art unsupervised methods even when a data set has been constructed without  ontological information, and when target words have not previously been seen in that position of a relationship in the training data.  Hence, we believe the supervised methods are learning characteristics of the underlying feature space which are generalisable to new words (inhabiting the same feature space).  

In future work, we intend to apply this approach to the problem of labelling the distributional neighbours found for a given word
 with specific semantic relations.

Finally, we believe that the data sets constructed from WordNet, which we publish alongside this paper,  can be used as a useful benchmark in evaluating future advances in this area, both for supervised and unsupervised methods.

\newpage

\bibliographystyle{acl}
\bibliography{JW2012}
\end{document}
